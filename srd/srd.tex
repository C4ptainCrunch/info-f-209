% !TEX encoding = UTF-8 Unicode
\documentclass[a4paper]{article}

\usepackage[utf8]{inputenc}
\usepackage[francais]{babel}

\title{Quiddich live : \\Software requirements document}
\author{Prénom 1 Nom 1 \and Prénom 2 Nom 2 \and Prénom 3 Nom 3 \and
Prénom 4 Nom 4 \and Prénom 5 Nom 5 \and Nikita Marchant}
\date{\today}

% Début du document
\begin{document}

\maketitle

\section{Introduction}
\subsection{But du projet}
\subsection{ Glossaire}
\subsection{Historique}
\section{Besoin d'utilisation}
\subsection{Exigence fonctionnelle}
\subsection{Exigence non fonctionnelle}
[A remplir en résumant le point \ref{enf}]
\subsection{Exigence de domaine}
\section{Besoin du système}
\subsection{Exigences fonctionnelles}
\subsection{Exigences non fonctionnelles}
\label{enf}

\begin{enumerate}
\item Le client et le serveur doivent être écrits en \textbf{C++}
\item Le client et le serveur doivent être portables et pouvoir fonctionner sur un système \textbf{UNIX} et une architecture x86
\item La machine hébergeant le client ainsi que le serveur doivent être en mesure de communiquer via un réseau capable de transporter des paquets \textbf{TCP/IP}
\item Le réseau décrit ci-dessus doit pouvoir une latence raisonnablement faible (c'est à dire de plus ou moins 200ms pour un aller retour)
\end{enumerate}


\subsection{Design et fonctionnement}
\section{Index}

% Les annexes
\appendix

\section{Premier annexe}
\section{Second annexe}

\section{Conclusion et discussion}
\addcontentsline{toc}{section}{Conclusion et discussion}

% Les différentes tables
\tableofcontents    % Table des matières
\listoffigures        % Liste des figures
\listoftables        % Liste des tableaux

% Fin du document
\end{document}