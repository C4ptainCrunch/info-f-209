% !TEX encoding = UTF-8 Unicode
\documentclass[a4paper]{article}

\usepackage[utf8]{inputenc}
\usepackage[francais]{babel}
\usepackage{hyperref}

% Glossaire:
\usepackage[acronym,xindy,toc]{glossaries}
\makeglossaries
% !TEX encoding = UTF-8 Unicode
% Glossaire
\newglossaryentry{joueur}
{
    name=joueur,
    description={Personnage fictif du jeux pouvant faire partie d'une équipe et jouant au quiditch.}
}
\newglossaryentry{equipe}
{
    name=équipe,
    description={Elle est composée de 7 \glspl{joueur} (un gardien, un attrapeur, deux batteurs et trois poursuive).}
}
\newglossaryentry{manager}
{
    name=manager,
    description={\Gls{utilisateur} du programme, il possède un \gls{club} et dois faire en sorte que celui-ci gagne de l'argent.}
}
\newglossaryentry{club}
{
    name=club,
    description={Infrastructure contenant une \gls{equipe} et des installations (stade, infirmerie,\ldots).}
}

\newglossaryentry{utilisateur}
{
    name=utilisateur,
    description={Personnage réel. C'est la personne qui joue au Quiditch manager.\\
    Elle est appelée en jeu \gls{manager}}
}

\newglossaryentry{enchere}
{
    name=enchère,
    description={Système permettant l'achat de \glspl{joueur}}
}
\newglossaryentry{serveur}
{
    name=serveur,
    description={Programme étant toujours en marche et connecté à internet et qui réceptionne les connections des \glspl{client}. Il a les données de tous les \glspl{utilisateur}.}
}
\newglossaryentry{client}
{
    name=client,
    description={Programme que l'\gls{utilisateur} lance pour se connecter au \gls{serveur}. Il permet à l'\gls{utilisateur} de jouer et peut être soit en ligne de commande, soit en mode graphique.}
}
\newglossaryentry{participant}
{
    name=participant,
    description={Nom donné à l'\gls{utilisateur} lors d'un tournoi.}
}


% Acronymes


\title{Quiddich live : \\Software requirements document}
\author{Prénom 1 Nom 1 \and Prénom 2 Nom 2 \and Prénom 3 Nom 3 \and
Romain Fontaine \and Tsotne Shonia \and Nikita Marchant}
\date{\today}

% Début du document
\begin{document}

\maketitle

\section{Introduction}
\subsection{But du projet}
\subsection{ Glossaire}
\printglossaries
\subsection{Historique}
\section{Besoin d'utilisation}
\subsection{Exigence fonctionnelle}
\begin{enumerate}
\item \textbf{Authentification} : L'utilisateur doit pouvoir s'enregistrer et se connecter sur le serveur avec un authentifiant unique et un mot de passe.
\item \textbf{Nouvelle carrière} : L'utilisateur doit pouvoir créer une nouvelle partie, lui conférant un nouveau \gls{club} rudimentaire.
\item \textbf{Management} : L'utilisateur doit pouvoir améliorer son \gls{club} en gérant son argent, son \gls{equipe}, ses infrastructures et son sponsoring.
\item \textbf{Tournois} : L'utilisateur doit pouvoir s'inscrire à un tournoi en ligne, auquel il devra participer à des matches contre d'autres joueurs.
\end{enumerate}
\subsection{Exigence non fonctionnelle}
[A remplir en résumant le point \ref{enf}]
\subsection{Exigence de domaine}
\begin{enumerate}
\item \textbf{Multijoueur} : Le serveur doit pouvoir acceuillir et gérer plusieurs connexions simultannées
\item \textbf{Monde persistent} : Le jeu continue d'évoluer, même en l'absence d'un ou plusieurs joueurs
\item \textbf{Connexion internet} : Une connexion internet est requise pour utilisé le programme client
\item \textbf{Protocole} : Le serveur et le client doivent communiquer avec le même protocole(?) afin de s'échanger des paquets de données
\item \textbf{Identification utilisateur} : Le serveur doit pouvoir d'identifier/authentifier un utilisateur afin de le connecter à son compte de jeu
\item \textbf{Vérification données} : Un système de vérification des données reçues (bits de parité, ...) doit être utilisé pour vérifier l'intégrité de celle-ci, et effectuer une correction si besoin
\end{enumerate}
\section{Besoin du système}
\subsection{Exigences fonctionnelles}

\begin{enumerate}
\item \textbf{Interface} : Le système doit fournir à l'utilisateur une interface jeu simple, complète et interactive, graphique et en console pour les deux phases de jeu.
\item \textbf{Représentation phase match} : Le système doit fournir à l'utilisateur une représentation du terrain (phase match) ovale, sous forme de cases hexagonales, où l'utilisateur pourra déplacer ses joueurs et / ou effectuer une action.
\item \textbf{Représentation phase management} : Le système doit fournir à l'utilisateur une représentation du \gls{club} (phase management), comprenant une vue d'ensemble sur son argent, ses rentrées, ses joueurs, ses infrastructures, les améliorations possibles, et l'avancement des tournois.
\item \textbf{Sauvegarde} : Le système doit sauvegarder tout changement effectué au \gls{club} durant le management ainsi qu'après une partie.
\item \textbf{Tournois} : Le système doit gérer les tournois en ligne, prenant en compte les divers cas possibles (forfait, absence, ...)
\end{enumerate}

\subsection{Exigences non fonctionnelles}
\label{enf}

\begin{enumerate}
\item Le client et le serveur doivent être écrits en \textbf{C++} seront compliés à l'aide de \textbf{gcc 4.2}
\item Le client et le serveur doivent être portables et pouvoir fonctionner sur un système \textbf{UNIX} et une architecture x86
\item La machine hébergeant le client ainsi que le serveur doivent être en mesure de communiquer en permanence via un réseau capable de transporter des paquets \textbf{TCP/IP}
\item Le réseau décrit ci-dessus doit pouvoir une latence raisonnablement faible (c'est à dire de plus ou moins 200ms pour un aller retour)
\item Les machines exécutant le client doivent être équipées d'un écran, d'un clavier et d'une souris et être capables d'afficher des images en mode graphique ainsi qu'au minimum 80 caractères en mode console. Elles devront aussi avoir au minimum disposer de 512Mb de mémoire vive ainsi que 500Mb d'espace disque.
\item La machine exécutant le serveur doit être capable de gérer une connexion ouverte constamment vers chaque client ainsi que de stocker l'entièreté des données du jeu en mémoire disque ainsi qu'une grande majorité en mémoire vive.
\end{enumerate}


\subsection{Design et fonctionnement}
\section{Index}

% Les annexes
\appendix

\section{Premier annexe}
\section{Second annexe}

\section{Conclusion et discussion}
\addcontentsline{toc}{section}{Conclusion et discussion}

% Les différentes tables
\tableofcontents    % Table des matières
\listoffigures        % Liste des figures
\listoftables        % Liste des tableaux

% Fin du document
\end{document}
