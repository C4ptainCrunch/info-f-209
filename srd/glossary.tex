% !TEX encoding = UTF-8 Unicode
% Glossaire
\newglossaryentry{joueur}
{
    name=joueur,
    description={Personnage fictif du jeux pouvant faire partie d'une équipe et jouant au quiditch.}
}
\newglossaryentry{equipe}
{
    name=équipe,
    description={Elle est composée de 7 \glspl{joueur} (un gardien, un attrapeur, deux batteurs et trois poursuive).}
}
\newglossaryentry{manager}
{
    name=manager,
    description={\Gls{utilisateur} du programme, il possède un \gls{club} et dois faire en sorte que celui-ci gagne de l'argent.}
}
\newglossaryentry{club}
{
    name=club,
    description={Infrastructure contenant une \gls{equipe} et des installations (stade, infirmerie,\ldots).}
}

\newglossaryentry{utilisateur}
{
    name=utilisateur,
    description={Personnage réel. C'est la personne qui joue au Quiditch manager.\\
    Elle est aussi appelée \gls{manager}}
}

\newglossaryentry{serveur}
{
    name=serveur,
    description={Programme étant toujours en marche et connecté à internet et qui réceptionne les connections des \glspl{client}. Il a les données de tous les \glspl{utilisateur}.}
}
\newglossaryentry{client}
{
    name=client,
    description={Programme que l'\gls{utilisateur} lance pour se connecter au \gls{serveur}. Il permet à l'\gls{utilisateur} de jouer et peut être soit en ligne de commande, soit en mode graphique.}
}


% Acronymes
