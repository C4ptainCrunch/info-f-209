% !TEX encoding = UTF-8 Unicode
\documentclass[a4paper,10pt]{article}
\usepackage[utf8]{inputenc}
\usepackage{listings}
\usepackage[francais]{babel}
\usepackage[T1]{fontenc}
\author{Présent: Tous (sauf Jérome)}
\title{PV réunion 5}
\date{19 décembre 2013}

% Début du document
\begin{document}
\maketitle
\part*{Introduction}
Dans "But du projet", rajouter les bases du Quidditch de façon synthétique (terrain oval, balles, ...)\\
Mettre référence possible "Quidditch à travers les âges" (?)\\
Système d'argent et tour par tour à rajouter dans l'intro\\
Délégué intro : Bruno
\part*{Glossaire}
Rajouter des termes pour le glossaire (participant)\\
Terme "arbre de tournois" OK\\
Délégué glossaire : Romain
\part*{Diagrammes}
ArgoUML permet les .eps -> mettre les diagrammes finis dans ce format et push\\
/!\textbackslash{} Attention, la figure doit expliquer brièvement chaque diagramme\\
Chacun met son diagramme dans le srd\\
Insérer les diagrammes au point 4\\
\part*{Autres}
Historique : une entrée (version finale est 1.0) -> Tsotne\\
Faire les classes vides (code c++) à partir du class diagram -> Cédric et Romain\\
(Créer dossier src, de tag [code], pour contenir les codes)\\
Exigences fonctionnelles -> Nikita\\
Orthographe -> Cédric (+ tout le mode)\\
Faire une table des matières à la fin sur une nouvelle page\\
\part*{Récapitulatif de la répartition des tâches}
Nikita -> Exigences fonctionnelles\\
Romain -> Code des classes; Glossaire\\
Cédric -> Code des classes; Vérifications orthographiques\\
Tsotne -> Historique du srd\\
Bruno -> Intro "But du projet"\\
(Jérome -> néant, comme sa présence dans le projet)
\end{document}