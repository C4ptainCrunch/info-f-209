% !TEX encoding = UTF-8 Unicode
\documentclass[a4paper,10pt]{article}
\usepackage[utf8]{inputenc}
\usepackage{listings}
\usepackage[francais]{babel}
\usepackage[T1]{fontenc}
\author{Présent: Tous (sauf Jérome)}
\title{PV réunion 5}
\date{19 décembre 2013}

% Début du document
\begin{document}
\maketitle
Dans but projet -> les règles de Quidditch (les bases de façon synthétique, terrain oval, cf quidditch à travers) -> Bruno
Système d'argent et tour par tour à rajouter dans l'intro
Rajouter des termes pour le glossaire (participant)
Terme arbre de tournois? OK
Délégé glossaire : Romain
ArgoUML permet les .eps -> mettre les diagrammes finis dans ce format et push
=> attention présentation avec figure pour expliquer chaque diagramme
Chacun met son diagramme dans le srd
Historique -> une entrée (version finale est 1.0) -> Tsotne
Faire les classes vides (code c++) à partir du class diagram -> Cédric et Romain
(créer dossier src de tag [code])
Exigences fonctionnelles -> Nikita
Orthographe -> Cédric (+ tout le mode)
Insérer les diagrammes au point 4
Faire table des matières à la fin sur nouvelle page
\end{document}