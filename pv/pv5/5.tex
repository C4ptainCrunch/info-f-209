% !TEX encoding = UTF-8 Unicode
\documentclass[a4paper,10pt]{article}
\usepackage[utf8]{inputenc}
\usepackage{listings}
\usepackage[francais]{babel}
\usepackage[T1]{fontenc}
\author{Présent: Tous (sauf Jérome)}
\title{PV réunion 5}
\date{19 décembre 2013}

% Début du document
\begin{document}
\maketitle
Dans but projet -> les règles de Quidditch (les bases de façon synthétique, terrain oval, cf quidditch à travers)
Rajouter des termes pour le glossaire (participant)
Terme arbre des matchs?
Système d'argent et tour par tour à rajouter dans l'intro
ArgoUML permet les .eps -> mettre les diagrammes finis dans ce format et push
=> attention présentation avec figure pour expliquer chaque diagramme
Chacun met son diagramme dans le srd
Délégé glossaire : Romain
Historique -> une entrée (version finale est 1.0)
Tsotne et Bruno -> rajouter les figures dans le srd
Faire les classes vides (code c++) à partir du class diagram -> Cédric et Romain
(créer dossier code)
Exigences fonctionnelles -> NikitA
Orthographe -> Cédric (+ tout le mode)

\end{document}